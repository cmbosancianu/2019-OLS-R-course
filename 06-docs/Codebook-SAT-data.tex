\documentclass[a4paper, 11pt]{article}
\usepackage[top=1in, bottom=1in, left=1in, right=1in]{geometry}
\usepackage[protrusion=true,expansion=true]{microtype} % Better typography
\usepackage{hyperref}
\usepackage[english]{babel}
\usepackage{palatino}
\usepackage[T1]{fontenc} % Required for accented characters
\makeatletter

\renewcommand{\maketitle}{
  \begin{flushleft}
    {\huge\@title}\\
    \vspace{10pt}
    {\large\@author}\\
    \vspace{10pt}
    {\@date}
    \vspace{40pt}
  \end{flushleft}
}

\title{\textbf{Codebook for SAT data}}
\author{\textsc{Constantin Manuel Bosancianu}\\ % Author
  \href{mailto:manuel.bosancianu@outlook.com}{manuel.bosancianu@outlook.com}\\
  \textit{Wissenschaftszentrum Berlin}\\
  \textit{Institutions and Political Inequality}} % Institution
\date{\today} % Date

\begin{document}
\maketitle

The data contains information on the average SAT score in each US state, collected around 1990-1991 in the United States. The data set is taken from Lawrence C. Hamilton, ``Statistics with STATA 9'', Duxbury Press, 2009. The original data set had a few more variables, but for the sake of keeping things manageable, I excluded them from the data.

The following variables can be found in the data:

\begin{enumerate}
\item \texttt{state}: state name;
\item \texttt{region}: geographic region in the United States;
\item \texttt{pop}: population of the state (information from the 1990 Census);
\item \texttt{area}: land area, in square miles;
\item \texttt{density}: population density (number of people per square mile);
\item \texttt{metro}: metropolitan area population, as percentage of the entire state population;
\item \texttt{waste}: per capita solid waste, measured in tons;
\item \texttt{energy}: per capita energy consumption, measured in BTUs;
\item \texttt{toxic}: per capita toxins released in the environments, measured in pounds;
\item \texttt{green}: per capita greenhouse gas emissions, measured in tons;
\item \texttt{csat}: average composite SAT score (verbal plus math);
\item \texttt{vsat}: average verbal SAT score;
\item \texttt{msat}: average mathematics SAT score;
\item \texttt{percent}: percentage of high school graduates taking the SAT;
\item \texttt{expense}: expenditure per student on primary and secondary education, measured in USD;
\item \texttt{income}: median household income, expressed in 1,000s of USD;
\item \texttt{high}: percentage of adults with a high school diploma;
\item \texttt{college}: percentage of adults with a college degree.
\end{enumerate}

\end{document}