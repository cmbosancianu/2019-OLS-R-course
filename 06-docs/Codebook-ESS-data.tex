\documentclass[a4paper, 11pt]{article}
\usepackage[top=1in, bottom=1in, left=1in, right=1in]{geometry}
\usepackage[protrusion=true,expansion=true]{microtype} % Better typography
\usepackage{hyperref}
\usepackage[english]{babel}
\usepackage{palatino}
\usepackage[T1]{fontenc} % Required for accented characters
\makeatletter

\renewcommand{\maketitle}{
  \begin{flushleft}
    {\Large\@title}\\
    \vspace{10pt}
    {\large\@author}\\
    \vspace{10pt}
    {\@date}
    \vspace{40pt}
  \end{flushleft}
}

\title{\textbf{Codebook for ESS UK data}}
\author{\textsc{Constantin Manuel Bosancianu}\\ % Author
  \href{mailto:manuel.bosancianu@outlook.com}{manuel.bosancianu@outlook.com}\\
  \textit{Wissenschaftszentrum Berlin f\"{u}r Sozialforschung\\
Institutions and Political Inequality}} % Institution
\date{\today} % Date

\begin{document}
\maketitle

The data is part of Round 7 of the European Social Survey, and contains only observations for Great Britain.

The following variables can be found in the data:

\begin{enumerate}
\item \texttt{stfdem}: satisfaction with the way democracy works in Great Britain, from 0 (``extremely dissatisfied'') to 1 (``extremely satisfied''); 
\item \texttt{male}: respondent is male (1=yes; 0=no);
\item \texttt{agea}: age of respondent, in years (ranging from 15 to 105);
\item \texttt{eduyrs}: years of full-time education;
\item \texttt{mbtru}: membership in trade union (1 = yes, now or at some point in the past; 0=no, never);
\item \texttt{hinctnta}: household income, after tax and contributions (1-10 scale, where 1 is the lowest decile, and 10 is the highest decile);
\item \texttt{ppltrst}: trust in other people, ranging from 0 (``can't be too careful'') to 10 (``most can be trusted'');
\item \texttt{age15}: age rescaled by subtracting 15 from each value (ranging from 0 to 90).
\item \texttt{clsprty}: closeness to any political party (1=yes; 0=no).
\end{enumerate}

\end{document}